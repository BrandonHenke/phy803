\documentclass[tikz]{standalone}
\usepackage{tikz-feynman}
\begin{document}
\begin{tikzpicture}
	% % FIRST DIAGRAM
	% \begin{feynman}
	% \vertex (v1);
	% \vertex [above left=of v1] (i0)  {$e^-$};
	% \vertex [below left=of v1] (i1) {$e^+$};
	% \vertex [right=of v1] (v2);
	% \vertex [above right=of v2] (f0) {$\bar{q}$};
	% \vertex [below right=of v2] (f1) {$q$};
	% \diagram* {
	% (i0) -- [fermion,momentum = $p_1$] (v1),
	% (i1) -- [anti fermion,momentum = $p_2$] (v1),
	% (v1) -- [photon, edge label = $\gamma$, momentum' = $q$] (v2),
	% (v2) -- [anti fermion,momentum = $p_3$] (f0),
	% (v2) -- [fermion,momentum = $p_4$] (f1),
	% };
	% \end{feynman}


	% % QCD DIAGRAM FOR PROBLEM 2
	% \begin{feynman}
	% 	\vertex (v1);
	% 	\vertex [above left=of v1] (i0)  {$q$};
	% 	\vertex [below left=of v1] (i1) {$\bar{q}$};
	% 	\vertex [right=of v1] (v2);
	% 	\vertex [above right=of v2] (f0) {$t$};
	% 	\vertex [below right=of v2] (f1) {$\bar{t}$};
	% 	\diagram* {
	% 	(i0) -- [fermion, edge label' = $i$,momentum = $p_1$] (v1),
	% 	(i1) -- [anti fermion, edge label' = $\bar{k}$, momentum = $p_2$] (v1),
	% 	(v1) -- [gluon, edge label = $g$, momentum' = $q$] (v2),
	% 	(v2) -- [anti fermion, edge label' = $j$, momentum = $p_3$] (f0),
	% 	(v2) -- [fermion, edge label' = $\bar{l}$, momentum = $p_4$] (f1),
	% 	};
	% \end{feynman}

	% % QED DIAGRAM FOR PROBLEM 2
	% \begin{feynman}
	% 	\vertex (v1);
	% 	\vertex [above left=of v1] (i0)  {$q$};
	% 	\vertex [below left=of v1] (i1) {$\bar{q}$};
	% 	\vertex [right=of v1] (v2);
	% 	\vertex [above right=of v2] (f0) {$t$};
	% 	\vertex [below right=of v2] (f1) {$\bar{t}$};
	% 	\diagram* {
	% 	(i0) -- [fermion, momentum = $p_1$] (v1),
	% 	(i1) -- [anti fermion, momentum = $p_2$] (v1),
	% 	(v1) -- [photon, edge label = $\gamma$, momentum' = $q$] (v2),
	% 	(v2) -- [anti fermion, momentum = $p_3$] (f0),
	% 	(v2) -- [fermion, momentum = $p_4$] (f1),
	% 	};
	% \end{feynman}
\end{tikzpicture}
\end{document}